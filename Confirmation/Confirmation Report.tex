\documentclass[12pt, titlepage]{article}
\usepackage{cite}
\usepackage{amsbsy}
\usepackage[a4paper, total={6in, 8in}, left=40mm]{geometry}
\usepackage{graphicx}
\usepackage{multirow}
\usepackage{booktabs}
\usepackage{wasysym}
\usepackage{float}
\usepackage{setspace}
\usepackage[euler]{textgreek}
\usepackage{textcomp}
\doublespacing

\setlength{\parskip}{1em}
\setlength{\parindent}{4em}
\graphicspath{ {./Images/} }
\title{Comparison of otolith derived field metabolic rates of six species of myctophids (Family Myctophidae) from the Scotia Sea, Southern Ocean\\
\large{Confirmation Report}}
\author{Sarah R. Alewijnse}     

\begin{document}
\maketitle
\tableofcontents
\pagebreak

\section{Abstract}

\section{Introduction}

Since the industrial revolution, an estimated third of the anthropogenic carbon dioxide (CO\textsubscript{2}) has been absorbed into the oceans. % cite
Despite being the fourth largest of the five major oceans, the Southern Ocean accounts for approximately 40\% of this carbon absorption; more than any other ocean. % cite
In the epipelagic zone, carbon can readily return to the atmosphere through the ocean-atmosphere CO\textsubscript{2} exchange.
However, below the remineralisation depth, carbon is effectively sequestered. 
Recent studies highlight the important role that mesopelagic fishes play in actively transporting carbon below the remineralisation depth. % cite various
Mesopelagic fishes which undertake diel vertical migration (DVM),  ingest carbon from the epipelagic by predating on zooplankton in the surface waters at night, and returning to the deep during the daylight, where they export carbon through respiration, egestion, excretion and mortality. 
Non-migratory mesopelagic fishes which reside below the remineralisation depth contribute to carbon sequestration by consuming migratory zooplakton. % Davidson
Alongside the recognition of this important role that mesopelagic fishes play in the ocean CO\textsubscript{2} sink, demand for fishmeal to sustain aquaculture is leading to interest in exploiting mesopelagic fishes, myctophids (family Myctophidae) in particular.
Myctophids are among the most abundant fishes in the oceans, and remain relatively unexploited up until now. % Catul Davison

The two issues of climate change and food security mean that understanding the role myctophids play in the biological carbon pump is especially crucial at this time.
Myctophids are the dominant mesopelagic fishes in the Scotia Sea, a highly productive area of the Southern Ocean. % Catul, Collins et al. 2012 (look for others) 
Scotia Sea myctophids are estimated to contribute 0.05 to 0.28 mg C m\textsuperscript{-2} d\textsuperscript{-1} to active carbon flux, equivalent to 9 to 47\% of gravitational particulate organic carbon flux in the same area. % Belcher
This estimate is obtained from myctophid metabolic rate, which is predicted from the regression equation:

\begin{equation}
Ln(R_{WM}) = a_{0} + a_{1} \times Ln(WM) + a_{2} \times T
\end{equation}

\noindent Where $WM$ is wet mass (g),  $R_{WM}$ is wet mass-specific metabolic rate (\textmu l O\textsubscript{2} mg WM\textsuperscript{-1} h\textsuperscript{-1}), and $T$ is temperature (\textdegree C).
This equation is parameterised based five studies of myctophid respiration rate, measured through either respirometry or electron transport system enzyme activity (ETS). % Cite studies
Myctophids are delicate fishes, and are most often dead or damaged on landing.
Consequently, fishes which are subjected to respirometry are likely to be stressed, giving an artificially high measure of metabolic rate.
Additionally, the most active fish in a catch are often selected for respirometry, which potentially biases measures towards those fish with higher metabolic rates. % Torres & Somero

Respirometry measures the standard or resting metabolic rate of a fish (SMR); the minimum metabolic rate of a resting organism, in a post-absorptive state. % Treberg
ETS measures the respiration potential of a sample of tissue, which is then converted to whole organism SMR using calibration curves of ETS and respirometry. % Cammen
A more useful measure of metabolic rate in the context of carbon modelling is field metabolic rate (FMR).
FMR is the time-averaged energy expenditure of an organism, free-ranging in its natural habitat.
FMR includes energy expended on basal costs, as with SMR, but also incorporates the thermic effect of food (also called specific dynamic action), as well as energy used for growth, reproduction, movement, egestion and excretion. % Treberg Chung

In this study, I used the stable isotope composition of carbon (\textdelta \textsuperscript{13}C) to calculate the proportion of metabolic carbon in the blood ($M$), a proxy for FMR % Chung & Trueman papers
for six species of myctophids common in the Scotia Sea. % Collins 2012, 2008 and Pusch et al. 2004
The aims of this study are to investigate (1) whether $M$ varies between species and (2) whether $M$ correlates signficantly with body weight (g) and temperature (\textdegree C).

\section{Methods}

\subsection{Otolith Preparation}

Otoliths, are structures made of calcium carbonate (usually aragonite) and an organic matrix in the inner ears of fishes. 
Otoliths grow sequentially, and once laid down, are metabolically inert, making them ideal structures for retrospectively aquiring metabolic rate and temperature. % Campana, Chung
The sagittal otoliths used in this study were collected in the Scotia Sea between 1998 and 2016 on cruises of the RRS James Clark Ross. % map
Otoliths from six species of Scotia Sea myctophids were studied: \textit{Electrona antarctica} (n = 19), \textit{E. carlsbergi} (n = 17), \textit{Gymnoscopelus braueri} (n = 20), \textit{G. nicholsi} (n = 13), \textit{Krefftichthys anderssoni} (n = 20) and \textit{Protomyctophum bolini} (n = 20). 
Each otolith was cleaned in water, allowed to dry and mounted onto a backing plate (Struers EpoFix resin).
The outer surface of the otolith (100-200 \textmu m depth) was milled using an ESI NewWave Micromill, to ensure that the otolith material laid down most closely to the time of capture was analysed.
Where the otoliths were too small to be milled ($<$1mm diameter - all \textit{K. anderssoni} and some \textit{P. bolini}), they were crushed to obtain powder for stable isotope analysis.

\subsection{Stable Isotope Analysis}

\textdelta \textsuperscript{13}C and \textdelta \textsuperscript{18}O of otolith aragonite were analysed using a Thermo Scientific Kiel IV Carbonate device, coupled with an MAT253 isotope ratio mass spectometer.
Stable isotope values are expressed permil (\permil) relative to Vienna Pee Dee Belemnite, according to the equation:

\begin{equation}
\delta X^{H} = \frac{R_{sample} - R_{standard}}{R_{standard}} \times 1000
\end{equation}

Where $X$ is the element, $^{H}$ is the mass of the heavy isotope of the element and $R$ is the ratio of the heavy to light isotope. % Fry

\subsection{Metabolic Carbon ($M$)}

The carbon in otolith aragonite is derived from carbon in the fish's blood, which is itself made of two main components. The first is dissolved inorganic carbon, ingested from the surrounding seawater. 
The second is metabolic carbon, which is produced during cellular respiration, and contains the carbon from the fish's diet. 
These two sources of carbon have isotopically distinct \textdelta \textsuperscript{13}C values, with \textdelta \textsuperscript{13}C from metabolic carbon being approximately 15$\permil$ higher than \textdelta \textsuperscript{13}C for dissolved inorganic carbon in most cases. % Chung 2019b
Based on this isotopic distinction, a simple mixing model can be used to calculate the proportion of metabolic carbon in the blood, $M$:

\begin{equation}
M = \frac{\delta^{13}C_{oto}-\delta^{13}C_{DIC-SW}}{\delta^{13}C_{diet}-\delta^{13}C_{DIC-SW}} + e_{total}
\end{equation}

\noindent Where $\delta^{13}C_{oto}$ is the \textdelta \textsuperscript{13}C value of the fish's otolith, $\delta^{13}C_{DIC-SW}$ is the value for \textdelta \textsuperscript{13}C of dissolved inorganic carbon (DIC) ingested by the fish through seawater, $\delta^{13}C_{diet}$ is the \textdelta \textsuperscript{13}C of the diet, and $e_{total}$ is the isotopic fractionation from DIC to blood, blood to endolymph, and endolymph to otolith.

In this study, parameters were allowed to vary across normal distributions with means, standard distributions, and minimum and maximum values set according to the relevant literature (see Appendix X). % set appendix number and cite each paper
$\delta^{13}C_{DIC-SW}$ was set using capture location and corrected for the Suess effect, from the model by Tagliabue \& Bopp, 2007.
$\delta^{13}C_{diet}$ was set using \textdelta \textsuperscript{13}C of phytoplankton (set by catch location) from the model by Magozzi \textit{et al.} 2017, trophic levels from FishBase and the trophic enrichment factor for carbon from DeNiro \& Epstein, 1978.
$e$ was set to the value from Solomon \textit{et al.} 2006.

\subsection{Temperature}

\textdelta \textsuperscript{18}O of otolith aragonite can be used to estimate the temperature of the water a fish was living in when that agragonite was precipitated. % Patterson et al. 1993, Thorrold et al. 1997
Temperature is estimated using the following equation:

\begin{equation}
T = \frac{(\delta^{18}O_{oto} - \delta^{18}O_{SW}) - a}{b}
\end{equation}

\noindent Where $\delta^{18}O_{oto}$ is the \textdelta \textsuperscript{18}O of the otolith, $\delta^{18}O_{SW}$ is the \textdelta \textsuperscript{18}O of the ambient seawater, and a and b are parameters, set in this study according to H{\o}ie \textit{et al.} 2004. $\delta^{18}O_{SW}$  is set according to Schmidt \textit{et al.} 1999, by location and depth. 
As with $M$, all parameters were allowed to vary across normal distributions (see Appendix X). % set appendix number and cite each paper

\subsection{Statistical Analyses}

All statistical analyses were carried out in R (version 3.4.4). 

\section{Results}

\section{Discussion}

\section{Conclusion}

\section{Thesis Plan and Gantt Chart}

\section{Acknowledgements}

\end{document}