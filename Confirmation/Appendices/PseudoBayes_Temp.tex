\documentclass[]{article}
\usepackage{lmodern}
\usepackage{amssymb,amsmath}
\usepackage{ifxetex,ifluatex}
\usepackage{fixltx2e} % provides \textsubscript
\ifnum 0\ifxetex 1\fi\ifluatex 1\fi=0 % if pdftex
  \usepackage[T1]{fontenc}
  \usepackage[utf8]{inputenc}
\else % if luatex or xelatex
  \ifxetex
    \usepackage{mathspec}
  \else
    \usepackage{fontspec}
  \fi
  \defaultfontfeatures{Ligatures=TeX,Scale=MatchLowercase}
\fi
% use upquote if available, for straight quotes in verbatim environments
\IfFileExists{upquote.sty}{\usepackage{upquote}}{}
% use microtype if available
\IfFileExists{microtype.sty}{%
\usepackage{microtype}
\UseMicrotypeSet[protrusion]{basicmath} % disable protrusion for tt fonts
}{}
\usepackage[margin=1in]{geometry}
\usepackage{hyperref}
\hypersetup{unicode=true,
            pdfborder={0 0 0},
            breaklinks=true}
\urlstyle{same}  % don't use monospace font for urls
\usepackage{color}
\usepackage{fancyvrb}
\newcommand{\VerbBar}{|}
\newcommand{\VERB}{\Verb[commandchars=\\\{\}]}
\DefineVerbatimEnvironment{Highlighting}{Verbatim}{commandchars=\\\{\}}
% Add ',fontsize=\small' for more characters per line
\usepackage{framed}
\definecolor{shadecolor}{RGB}{248,248,248}
\newenvironment{Shaded}{\begin{snugshade}}{\end{snugshade}}
\newcommand{\AlertTok}[1]{\textcolor[rgb]{0.94,0.16,0.16}{#1}}
\newcommand{\AnnotationTok}[1]{\textcolor[rgb]{0.56,0.35,0.01}{\textbf{\textit{#1}}}}
\newcommand{\AttributeTok}[1]{\textcolor[rgb]{0.77,0.63,0.00}{#1}}
\newcommand{\BaseNTok}[1]{\textcolor[rgb]{0.00,0.00,0.81}{#1}}
\newcommand{\BuiltInTok}[1]{#1}
\newcommand{\CharTok}[1]{\textcolor[rgb]{0.31,0.60,0.02}{#1}}
\newcommand{\CommentTok}[1]{\textcolor[rgb]{0.56,0.35,0.01}{\textit{#1}}}
\newcommand{\CommentVarTok}[1]{\textcolor[rgb]{0.56,0.35,0.01}{\textbf{\textit{#1}}}}
\newcommand{\ConstantTok}[1]{\textcolor[rgb]{0.00,0.00,0.00}{#1}}
\newcommand{\ControlFlowTok}[1]{\textcolor[rgb]{0.13,0.29,0.53}{\textbf{#1}}}
\newcommand{\DataTypeTok}[1]{\textcolor[rgb]{0.13,0.29,0.53}{#1}}
\newcommand{\DecValTok}[1]{\textcolor[rgb]{0.00,0.00,0.81}{#1}}
\newcommand{\DocumentationTok}[1]{\textcolor[rgb]{0.56,0.35,0.01}{\textbf{\textit{#1}}}}
\newcommand{\ErrorTok}[1]{\textcolor[rgb]{0.64,0.00,0.00}{\textbf{#1}}}
\newcommand{\ExtensionTok}[1]{#1}
\newcommand{\FloatTok}[1]{\textcolor[rgb]{0.00,0.00,0.81}{#1}}
\newcommand{\FunctionTok}[1]{\textcolor[rgb]{0.00,0.00,0.00}{#1}}
\newcommand{\ImportTok}[1]{#1}
\newcommand{\InformationTok}[1]{\textcolor[rgb]{0.56,0.35,0.01}{\textbf{\textit{#1}}}}
\newcommand{\KeywordTok}[1]{\textcolor[rgb]{0.13,0.29,0.53}{\textbf{#1}}}
\newcommand{\NormalTok}[1]{#1}
\newcommand{\OperatorTok}[1]{\textcolor[rgb]{0.81,0.36,0.00}{\textbf{#1}}}
\newcommand{\OtherTok}[1]{\textcolor[rgb]{0.56,0.35,0.01}{#1}}
\newcommand{\PreprocessorTok}[1]{\textcolor[rgb]{0.56,0.35,0.01}{\textit{#1}}}
\newcommand{\RegionMarkerTok}[1]{#1}
\newcommand{\SpecialCharTok}[1]{\textcolor[rgb]{0.00,0.00,0.00}{#1}}
\newcommand{\SpecialStringTok}[1]{\textcolor[rgb]{0.31,0.60,0.02}{#1}}
\newcommand{\StringTok}[1]{\textcolor[rgb]{0.31,0.60,0.02}{#1}}
\newcommand{\VariableTok}[1]{\textcolor[rgb]{0.00,0.00,0.00}{#1}}
\newcommand{\VerbatimStringTok}[1]{\textcolor[rgb]{0.31,0.60,0.02}{#1}}
\newcommand{\WarningTok}[1]{\textcolor[rgb]{0.56,0.35,0.01}{\textbf{\textit{#1}}}}
\usepackage{graphicx,grffile}
\makeatletter
\def\maxwidth{\ifdim\Gin@nat@width>\linewidth\linewidth\else\Gin@nat@width\fi}
\def\maxheight{\ifdim\Gin@nat@height>\textheight\textheight\else\Gin@nat@height\fi}
\makeatother
% Scale images if necessary, so that they will not overflow the page
% margins by default, and it is still possible to overwrite the defaults
% using explicit options in \includegraphics[width, height, ...]{}
\setkeys{Gin}{width=\maxwidth,height=\maxheight,keepaspectratio}
\IfFileExists{parskip.sty}{%
\usepackage{parskip}
}{% else
\setlength{\parindent}{0pt}
\setlength{\parskip}{6pt plus 2pt minus 1pt}
}
\setlength{\emergencystretch}{3em}  % prevent overfull lines
\providecommand{\tightlist}{%
  \setlength{\itemsep}{0pt}\setlength{\parskip}{0pt}}
\setcounter{secnumdepth}{0}
% Redefines (sub)paragraphs to behave more like sections
\ifx\paragraph\undefined\else
\let\oldparagraph\paragraph
\renewcommand{\paragraph}[1]{\oldparagraph{#1}\mbox{}}
\fi
\ifx\subparagraph\undefined\else
\let\oldsubparagraph\subparagraph
\renewcommand{\subparagraph}[1]{\oldsubparagraph{#1}\mbox{}}
\fi

%%% Use protect on footnotes to avoid problems with footnotes in titles
\let\rmarkdownfootnote\footnote%
\def\footnote{\protect\rmarkdownfootnote}

%%% Change title format to be more compact
\usepackage{titling}

% Create subtitle command for use in maketitle
\newcommand{\subtitle}[1]{
  \posttitle{
    \begin{center}\large#1\end{center}
    }
}

\setlength{\droptitle}{-2em}
  \title{}
  \pretitle{\vspace{\droptitle}}
  \posttitle{}
  \author{}
  \preauthor{}\postauthor{}
  \date{}
  \predate{}\postdate{}


\begin{document}

\begin{Shaded}
\begin{Highlighting}[]
\KeywordTok{library}\NormalTok{(tidyverse)}
\KeywordTok{library}\NormalTok{(HDInterval)}
\KeywordTok{library}\NormalTok{(truncnorm)}

\NormalTok{Temp_Vals <-}\StringTok{ }\ControlFlowTok{function}\NormalTok{(d18O, d18O_sd, }
                             \CommentTok{# d18O_otolith values and SD}
\NormalTok{                             d18O_water, d18O_water_sd, }
                             \CommentTok{# d18O_SW values and SD}
\NormalTok{                             d18O_water_min, d18O_water_max, }
                             \CommentTok{# Global min and max for d18O_SW}
\NormalTok{                             reps)\{ }
                             \CommentTok{# Number of repetitions when sampling from distributions}
  
  \CommentTok{# Calculate distributions of component values with set.seeds to ensure reproducibility}
  \KeywordTok{set.seed}\NormalTok{(d18O)}
\NormalTok{  dist_d18O <-}\StringTok{ }\KeywordTok{rnorm}\NormalTok{(reps, d18O, d18O_sd)}
  \KeywordTok{set.seed}\NormalTok{(d18O_water)}
\NormalTok{  dist_d18O_water <-}\StringTok{ }\KeywordTok{rtruncnorm}\NormalTok{(reps, d18O_water_min, d18O_water_max, d18O_water, }
\NormalTok{                                d18O_water_sd)}
  \CommentTok{# Set parameter values from Hoie et al. 2004}
  \KeywordTok{set.seed}\NormalTok{(}\FloatTok{3.9}\NormalTok{)}
\NormalTok{  dist_param_}\DecValTok{1}\NormalTok{ <-}\StringTok{ }\KeywordTok{rnorm}\NormalTok{(reps, }\FloatTok{3.90}\NormalTok{, }\FloatTok{0.24}\NormalTok{)}
  \KeywordTok{set.seed}\NormalTok{(}\OperatorTok{-}\FloatTok{0.20}\NormalTok{)}
\NormalTok{  dist_param_}\DecValTok{2}\NormalTok{ <-}\StringTok{ }\KeywordTok{rnorm}\NormalTok{(reps, }\FloatTok{-0.20}\NormalTok{, }\FloatTok{0.019}\NormalTok{)}
  
  \CommentTok{# Calculate temperature}
\NormalTok{  dist_d18 <-}\StringTok{ }\NormalTok{dist_d18O }\OperatorTok{-}\StringTok{ }\NormalTok{dist_d18O_water}
\NormalTok{  dist_temp <-}\StringTok{ }\NormalTok{(dist_d18 }\OperatorTok{-}\StringTok{ }\NormalTok{dist_param_}\DecValTok{1}\NormalTok{)}\OperatorTok{/}\NormalTok{dist_param_}\DecValTok{2}
  
  \CommentTok{# Get maximum density of temperature}
\NormalTok{  max_dens <-}\StringTok{ }\KeywordTok{which.max}\NormalTok{(}\KeywordTok{density}\NormalTok{(dist_temp)}\OperatorTok{$}\NormalTok{y)}
\NormalTok{  temp <-}\StringTok{ }\KeywordTok{density}\NormalTok{(dist_temp)}\OperatorTok{$}\NormalTok{x[max_dens]}
  
  \CommentTok{# Get minimum, maximum and sd of temperature}
\NormalTok{  min_temp <-}\StringTok{ }\KeywordTok{min}\NormalTok{(dist_temp)}
\NormalTok{  max_temp <-}\StringTok{ }\KeywordTok{max}\NormalTok{(dist_temp)}
\NormalTok{  sd_temp <-}\StringTok{ }\KeywordTok{sd}\NormalTok{(dist_temp)}
  
  \CommentTok{# Return results}
\NormalTok{  result <-}\StringTok{ }\KeywordTok{data.frame}\NormalTok{(temp, sd_temp, min_temp, max_temp)}
  \KeywordTok{return}\NormalTok{(result)}
\NormalTok{\}}
\end{Highlighting}
\end{Shaded}

\begin{Shaded}
\begin{Highlighting}[]
\CommentTok{# Read in and tidy data}
\KeywordTok{setwd}\NormalTok{(}\StringTok{"~/PhD/GitHub/mytophid-ears"}\NormalTok{)}
\NormalTok{myct <-}\StringTok{ }\KeywordTok{read.csv}\NormalTok{(}\StringTok{"Data/Myctophids_Master.csv"}\NormalTok{)}
\NormalTok{myct <-}\StringTok{ }\KeywordTok{filter}\NormalTok{(myct, d13C }\OperatorTok{!=}\StringTok{ "NA"}\NormalTok{)}
\NormalTok{d18O_sd <-}\StringTok{ }\KeywordTok{sd}\NormalTok{(myct}\OperatorTok{$}\NormalTok{D18O_vals)}
\NormalTok{d18O_min <-}\StringTok{ }\KeywordTok{min}\NormalTok{(myct}\OperatorTok{$}\NormalTok{D18O_vals)}
\NormalTok{d18O_max <-}\StringTok{ }\KeywordTok{max}\NormalTok{(myct}\OperatorTok{$}\NormalTok{D18O_vals)}

\KeywordTok{with}\NormalTok{(myct[}\DecValTok{3}\NormalTok{,], }\CommentTok{# Fish number three in master data set}
     \KeywordTok{Temp_Vals}\NormalTok{(d18O, }\FloatTok{0.02}\NormalTok{, }\CommentTok{# SD from isotope data}
\NormalTok{                      D18O_vals, d18O_sd, }\CommentTok{# SD from data}
\NormalTok{                      d18O_min, d18O_max, }\CommentTok{# Min-max from data}
                      \DecValTok{10000}\NormalTok{))}
\end{Highlighting}
\end{Shaded}

\begin{verbatim}
##        temp  sd_temp  min_temp max_temp
## 1 0.1037534 1.230554 -4.258677 5.742518
\end{verbatim}

\begin{itemize}
\tightlist
\item
  temp = value at maximum density of distribution of temperature values.
\item
  sd\_temp = standard deviation of temperature distribution.
\item
  min\_temp = minimum value of temperature distribution.
\item
  max\_temp = maximum value of temperature distribution.
\end{itemize}


\end{document}
