\documentclass[]{article}
\usepackage{lmodern}
\usepackage{amssymb,amsmath}
\usepackage{ifxetex,ifluatex}
\usepackage{fixltx2e} % provides \textsubscript
\ifnum 0\ifxetex 1\fi\ifluatex 1\fi=0 % if pdftex
  \usepackage[T1]{fontenc}
  \usepackage[utf8]{inputenc}
\else % if luatex or xelatex
  \ifxetex
    \usepackage{mathspec}
  \else
    \usepackage{fontspec}
  \fi
  \defaultfontfeatures{Ligatures=TeX,Scale=MatchLowercase}
\fi
% use upquote if available, for straight quotes in verbatim environments
\IfFileExists{upquote.sty}{\usepackage{upquote}}{}
% use microtype if available
\IfFileExists{microtype.sty}{%
\usepackage{microtype}
\UseMicrotypeSet[protrusion]{basicmath} % disable protrusion for tt fonts
}{}
\usepackage[margin=1in]{geometry}
\usepackage{hyperref}
\hypersetup{unicode=true,
            pdfborder={0 0 0},
            breaklinks=true}
\urlstyle{same}  % don't use monospace font for urls
\usepackage{color}
\usepackage{fancyvrb}
\newcommand{\VerbBar}{|}
\newcommand{\VERB}{\Verb[commandchars=\\\{\}]}
\DefineVerbatimEnvironment{Highlighting}{Verbatim}{commandchars=\\\{\}}
% Add ',fontsize=\small' for more characters per line
\usepackage{framed}
\definecolor{shadecolor}{RGB}{248,248,248}
\newenvironment{Shaded}{\begin{snugshade}}{\end{snugshade}}
\newcommand{\AlertTok}[1]{\textcolor[rgb]{0.94,0.16,0.16}{#1}}
\newcommand{\AnnotationTok}[1]{\textcolor[rgb]{0.56,0.35,0.01}{\textbf{\textit{#1}}}}
\newcommand{\AttributeTok}[1]{\textcolor[rgb]{0.77,0.63,0.00}{#1}}
\newcommand{\BaseNTok}[1]{\textcolor[rgb]{0.00,0.00,0.81}{#1}}
\newcommand{\BuiltInTok}[1]{#1}
\newcommand{\CharTok}[1]{\textcolor[rgb]{0.31,0.60,0.02}{#1}}
\newcommand{\CommentTok}[1]{\textcolor[rgb]{0.56,0.35,0.01}{\textit{#1}}}
\newcommand{\CommentVarTok}[1]{\textcolor[rgb]{0.56,0.35,0.01}{\textbf{\textit{#1}}}}
\newcommand{\ConstantTok}[1]{\textcolor[rgb]{0.00,0.00,0.00}{#1}}
\newcommand{\ControlFlowTok}[1]{\textcolor[rgb]{0.13,0.29,0.53}{\textbf{#1}}}
\newcommand{\DataTypeTok}[1]{\textcolor[rgb]{0.13,0.29,0.53}{#1}}
\newcommand{\DecValTok}[1]{\textcolor[rgb]{0.00,0.00,0.81}{#1}}
\newcommand{\DocumentationTok}[1]{\textcolor[rgb]{0.56,0.35,0.01}{\textbf{\textit{#1}}}}
\newcommand{\ErrorTok}[1]{\textcolor[rgb]{0.64,0.00,0.00}{\textbf{#1}}}
\newcommand{\ExtensionTok}[1]{#1}
\newcommand{\FloatTok}[1]{\textcolor[rgb]{0.00,0.00,0.81}{#1}}
\newcommand{\FunctionTok}[1]{\textcolor[rgb]{0.00,0.00,0.00}{#1}}
\newcommand{\ImportTok}[1]{#1}
\newcommand{\InformationTok}[1]{\textcolor[rgb]{0.56,0.35,0.01}{\textbf{\textit{#1}}}}
\newcommand{\KeywordTok}[1]{\textcolor[rgb]{0.13,0.29,0.53}{\textbf{#1}}}
\newcommand{\NormalTok}[1]{#1}
\newcommand{\OperatorTok}[1]{\textcolor[rgb]{0.81,0.36,0.00}{\textbf{#1}}}
\newcommand{\OtherTok}[1]{\textcolor[rgb]{0.56,0.35,0.01}{#1}}
\newcommand{\PreprocessorTok}[1]{\textcolor[rgb]{0.56,0.35,0.01}{\textit{#1}}}
\newcommand{\RegionMarkerTok}[1]{#1}
\newcommand{\SpecialCharTok}[1]{\textcolor[rgb]{0.00,0.00,0.00}{#1}}
\newcommand{\SpecialStringTok}[1]{\textcolor[rgb]{0.31,0.60,0.02}{#1}}
\newcommand{\StringTok}[1]{\textcolor[rgb]{0.31,0.60,0.02}{#1}}
\newcommand{\VariableTok}[1]{\textcolor[rgb]{0.00,0.00,0.00}{#1}}
\newcommand{\VerbatimStringTok}[1]{\textcolor[rgb]{0.31,0.60,0.02}{#1}}
\newcommand{\WarningTok}[1]{\textcolor[rgb]{0.56,0.35,0.01}{\textbf{\textit{#1}}}}
\usepackage{graphicx,grffile}
\makeatletter
\def\maxwidth{\ifdim\Gin@nat@width>\linewidth\linewidth\else\Gin@nat@width\fi}
\def\maxheight{\ifdim\Gin@nat@height>\textheight\textheight\else\Gin@nat@height\fi}
\makeatother
% Scale images if necessary, so that they will not overflow the page
% margins by default, and it is still possible to overwrite the defaults
% using explicit options in \includegraphics[width, height, ...]{}
\setkeys{Gin}{width=\maxwidth,height=\maxheight,keepaspectratio}
\IfFileExists{parskip.sty}{%
\usepackage{parskip}
}{% else
\setlength{\parindent}{0pt}
\setlength{\parskip}{6pt plus 2pt minus 1pt}
}
\setlength{\emergencystretch}{3em}  % prevent overfull lines
\providecommand{\tightlist}{%
  \setlength{\itemsep}{0pt}\setlength{\parskip}{0pt}}
\setcounter{secnumdepth}{0}
% Redefines (sub)paragraphs to behave more like sections
\ifx\paragraph\undefined\else
\let\oldparagraph\paragraph
\renewcommand{\paragraph}[1]{\oldparagraph{#1}\mbox{}}
\fi
\ifx\subparagraph\undefined\else
\let\oldsubparagraph\subparagraph
\renewcommand{\subparagraph}[1]{\oldsubparagraph{#1}\mbox{}}
\fi

%%% Use protect on footnotes to avoid problems with footnotes in titles
\let\rmarkdownfootnote\footnote%
\def\footnote{\protect\rmarkdownfootnote}

%%% Change title format to be more compact
\usepackage{titling}

% Create subtitle command for use in maketitle
\newcommand{\subtitle}[1]{
  \posttitle{
    \begin{center}\large#1\end{center}
    }
}

\setlength{\droptitle}{-2em}
  \title{}
  \pretitle{\vspace{\droptitle}}
  \posttitle{}
  \author{}
  \preauthor{}\postauthor{}
  \date{}
  \predate{}\postdate{}


\begin{document}

\begin{Shaded}
\begin{Highlighting}[]
\KeywordTok{library}\NormalTok{(tidyverse)}
\KeywordTok{library}\NormalTok{(HDInterval)}
\KeywordTok{library}\NormalTok{(truncnorm)}

\NormalTok{M_val <-}\StringTok{ }\ControlFlowTok{function}\NormalTok{(d13C, d13C_sd, }
                  \CommentTok{# d13C_otolith value and SD}
\NormalTok{                  Year, }
                  \CommentTok{# Year of capture}
\NormalTok{                  reps, }
                  \CommentTok{# Number of repetitions when sampling from distributions}
\NormalTok{                  DIC_surf, DIC_surf_sd, }
                  \CommentTok{# d13C_DIC value and SD from capture location}
\NormalTok{                  DIC_min, DIC_max, }
                  \CommentTok{# Global d13C_DIC min and max}
\NormalTok{                  suess, suess_sd, }
                  \CommentTok{# Post-1970 Suess effect and SD from capture location}
\NormalTok{                  suess_min, suess_max, }
                  \CommentTok{# Global Suess effect min and max}
\NormalTok{                  suess_}\DecValTok{1970}\NormalTok{, suess_}\DecValTok{1970}\NormalTok{_sd, }
                  \CommentTok{# Pre-1970 Suess effect and SD}
\NormalTok{                  phyto, phyto_sd, }
                  \CommentTok{# d13C_phyto and SD from capture location}
\NormalTok{                  phyto_min, phyto_max, }
                  \CommentTok{# Global d13C_phyto min and max}
\NormalTok{                  phyto_suess, phyto_suess_sd, }
                  \CommentTok{# Suess effect for phytoplankton and SD}
\NormalTok{                  troph, troph_se,}
                  \CommentTok{# Trophic level and SE}
\NormalTok{                  troph_min, troph_max, }
                  \CommentTok{# Min and max trophic level for fishes}
\NormalTok{                  enrich, enrich_se, }
                  \CommentTok{# Trophic enrichment of d13C and SD}
\NormalTok{                  e_value)\{ }\CommentTok{# e (fractiontion) term}
  
  \CommentTok{# Calculate distributions of component values with set.seeds to ensure reproducibility}
  \KeywordTok{set.seed}\NormalTok{(d13C)}
\NormalTok{  dist_d13C <-}\StringTok{ }\KeywordTok{rnorm}\NormalTok{(reps, d13C, d13C_sd)}
  \KeywordTok{set.seed}\NormalTok{(DIC_surf)}
\NormalTok{  dist_DIC_surf <-}\StringTok{ }\KeywordTok{rtruncnorm}\NormalTok{(reps, DIC_min, DIC_max, DIC_surf, DIC_surf_sd)}
  \KeywordTok{set.seed}\NormalTok{(suess)}
\NormalTok{  dist_suess <-}\StringTok{ }\KeywordTok{rtruncnorm}\NormalTok{(reps, suess_min, suess_max, suess, suess_sd)}
  \KeywordTok{set.seed}\NormalTok{(suess_}\DecValTok{1970}\NormalTok{)}
\NormalTok{  dist_suess_}\DecValTok{1970}\NormalTok{ <-}\StringTok{ }\KeywordTok{rtruncnorm}\NormalTok{(reps, suess_min, suess_max, suess_}\DecValTok{1970}\NormalTok{, suess_}\DecValTok{1970}\NormalTok{_sd)}
  \KeywordTok{set.seed}\NormalTok{(phyto)}
\NormalTok{  dist_phyto <-}\StringTok{ }\KeywordTok{rtruncnorm}\NormalTok{(reps, phyto_min, phyto_max, phyto, phyto_sd)}
  \KeywordTok{set.seed}\NormalTok{(phyto_suess)}
\NormalTok{  dist_phyto_suess <-}\StringTok{ }\KeywordTok{rtruncnorm}\NormalTok{(reps, suess_min, suess_max, phyto_suess, phyto_suess_sd)}
  \KeywordTok{set.seed}\NormalTok{(troph)}
\NormalTok{  dist_troph <-}\StringTok{ }\KeywordTok{rtruncnorm}\NormalTok{(reps, troph_min, troph_max, troph, troph_se)}
  \KeywordTok{set.seed}\NormalTok{(enrich)}
\NormalTok{  dist_enrich <-}\StringTok{ }\KeywordTok{rnorm}\NormalTok{(reps, enrich, enrich_se)}
  \KeywordTok{set.seed}\NormalTok{(}\KeywordTok{Sys.time}\NormalTok{())}
  
  \CommentTok{# Calculate d13C_DIC, correcting for the Suess effect}
\NormalTok{  dist_DIC <-}\StringTok{ }\ControlFlowTok{if}\NormalTok{(Year }\OperatorTok{<}\StringTok{ }\DecValTok{1970}\NormalTok{)\{}
\NormalTok{    dist_DIC_surf}\OperatorTok{-}\NormalTok{(dist_suess_}\DecValTok{1970}\OperatorTok{*}\NormalTok{((}\DecValTok{1990}\OperatorTok{-}\NormalTok{Year)}\OperatorTok{/}\DecValTok{10}\NormalTok{))}
\NormalTok{  \} }\ControlFlowTok{else} \ControlFlowTok{if}\NormalTok{(Year }\OperatorTok{<}\StringTok{ }\DecValTok{1990}\NormalTok{)\{}
\NormalTok{    dist_DIC_surf}\OperatorTok{-}\NormalTok{(dist_suess}\OperatorTok{*}\NormalTok{((}\DecValTok{1990}\OperatorTok{-}\NormalTok{Year)}\OperatorTok{/}\DecValTok{10}\NormalTok{))}
\NormalTok{  \} }\ControlFlowTok{else}\NormalTok{ \{}
\NormalTok{    dist_DIC_surf}\OperatorTok{+}\NormalTok{(dist_suess}\OperatorTok{*}\NormalTok{((}\DecValTok{1990}\OperatorTok{-}\NormalTok{Year)}\OperatorTok{/}\DecValTok{10}\NormalTok{))}
\NormalTok{  \}}
  
  \CommentTok{# Calculate d13C_Phyto, correcting for the Suess effect}
\NormalTok{  dist_phyto_corr <-}\StringTok{ }\ControlFlowTok{if}\NormalTok{(Year }\OperatorTok{<}\StringTok{ }\DecValTok{1970}\NormalTok{)\{}
\NormalTok{    dist_phyto}\OperatorTok{-}\NormalTok{(dist_phyto_suess}\OperatorTok{*}\NormalTok{((}\DecValTok{2001}\OperatorTok{-}\NormalTok{Year)}\OperatorTok{/}\DecValTok{10}\NormalTok{))}
\NormalTok{  \} }\ControlFlowTok{else} \ControlFlowTok{if}\NormalTok{(Year }\OperatorTok{<}\StringTok{ }\DecValTok{2001}\NormalTok{)\{}
\NormalTok{    dist_phyto}\OperatorTok{-}\NormalTok{(dist_phyto_suess}\OperatorTok{*}\NormalTok{((}\DecValTok{2001}\OperatorTok{-}\NormalTok{Year)}\OperatorTok{/}\DecValTok{10}\NormalTok{))}
\NormalTok{  \} }\ControlFlowTok{else} \ControlFlowTok{if}\NormalTok{(Year }\OperatorTok{>}\StringTok{ }\DecValTok{2010}\NormalTok{)\{}
\NormalTok{    dist_phyto}\OperatorTok{+}\NormalTok{(dist_phyto_suess}\OperatorTok{*}\NormalTok{((}\DecValTok{2010}\OperatorTok{-}\NormalTok{Year)}\OperatorTok{/}\DecValTok{10}\NormalTok{))}
\NormalTok{  \} }\ControlFlowTok{else}\NormalTok{ \{}
\NormalTok{    dist_phyto}
\NormalTok{  \}}
  
  \CommentTok{# Calculate d13C_diet}
\NormalTok{  dist_diet <-}\StringTok{ }\NormalTok{dist_phyto_corr }\OperatorTok{+}\StringTok{ }\NormalTok{(dist_troph }\OperatorTok{-}\StringTok{ }\DecValTok{1}\NormalTok{) }\OperatorTok{*}\StringTok{ }\NormalTok{dist_enrich}
  
  \CommentTok{# Calculate M}
\NormalTok{  dist_M <-}\StringTok{ }\NormalTok{(dist_d13C }\OperatorTok{-}\StringTok{ }\NormalTok{dist_DIC) }\OperatorTok{/}\StringTok{ }\NormalTok{(dist_diet }\OperatorTok{-}\StringTok{ }\NormalTok{dist_DIC) }\OperatorTok{+}\StringTok{ }\NormalTok{e_value}
  
  \CommentTok{# Get maximum density value for M}
\NormalTok{  max_dens <-}\StringTok{ }\KeywordTok{which.max}\NormalTok{(}\KeywordTok{density}\NormalTok{(dist_M)}\OperatorTok{$}\NormalTok{y)}
\NormalTok{  M <-}\StringTok{ }\KeywordTok{density}\NormalTok{(dist_M)}\OperatorTok{$}\NormalTok{x[max_dens]}
  
  \CommentTok{# Get minimum, maximum and sd for M distribution}
\NormalTok{  min_M <-}\StringTok{ }\KeywordTok{min}\NormalTok{(dist_M)}
\NormalTok{  max_M <-}\StringTok{ }\KeywordTok{max}\NormalTok{(dist_M)}
\NormalTok{  sd_M <-}\StringTok{ }\KeywordTok{sd}\NormalTok{(dist_M)}
  
  \CommentTok{# Combine results}
\NormalTok{  result <-}\StringTok{ }\KeywordTok{data.frame}\NormalTok{(M, sd_M, min_M, max_M)}
  \KeywordTok{return}\NormalTok{(result)}
\NormalTok{\}}
\end{Highlighting}
\end{Shaded}

\begin{Shaded}
\begin{Highlighting}[]
\CommentTok{# Read in data}

\KeywordTok{setwd}\NormalTok{(}\StringTok{"~/PhD/GitHub/mytophid-ears"}\NormalTok{)}
\KeywordTok{load}\NormalTok{(}\StringTok{"Functions/M_Val.Rdata"}\NormalTok{)}
\NormalTok{myct <-}\StringTok{ }\KeywordTok{read.csv}\NormalTok{(}\StringTok{"Data/Myctophids_Master.csv"}\NormalTok{)}

\KeywordTok{with}\NormalTok{(myct[}\DecValTok{2}\NormalTok{,], }\CommentTok{# Fish number two in master data set}
     \KeywordTok{M_Val}\NormalTok{(d13C, }\FloatTok{0.02}\NormalTok{, }
           \CommentTok{# SD from isotope lab}
\NormalTok{           Year.x, }\DecValTok{10000}\NormalTok{,}
\NormalTok{           DIC, }\FloatTok{0.202}\NormalTok{, }\DecValTok{0}\NormalTok{, }\DecValTok{3}\NormalTok{, }
           \CommentTok{# From Tagliabue & Bopp, 2007}
\NormalTok{           Suess, }\FloatTok{0.202}\NormalTok{, }\FloatTok{-0.28}\NormalTok{, }\DecValTok{0}\NormalTok{, }
           \CommentTok{# From Tagliabue & Bopp, 2007. SD same as for DIC}
           \FloatTok{-0.07}\NormalTok{, }\FloatTok{0.202}\NormalTok{, }
           \CommentTok{# From Tagliabue & Bopp, 2007. SD same as for DIC}
\NormalTok{           Phyto, Phyto_sd, }\DecValTok{-31}\NormalTok{, }\FloatTok{-16.5}\NormalTok{, }
           \CommentTok{# From Magozzi et al. 2017. Based on values for Longhurst provinces}
           \FloatTok{-0.17}\NormalTok{, }\FloatTok{0.202}\NormalTok{, }
           \CommentTok{# From Clive - per comms. SD same as for DIC}
\NormalTok{           UseTroph, UseTrophSe, }\DecValTok{2}\NormalTok{, }\DecValTok{5}\NormalTok{, }
           \CommentTok{# From FishBase}
           \FloatTok{0.8}\NormalTok{, }\FloatTok{1.1}\NormalTok{, }
           \CommentTok{# From DeNiro & Epstein 1978}
           \DecValTok{0}\NormalTok{)) }\CommentTok{# From Solomon et al. 2006}
\end{Highlighting}
\end{Shaded}

\begin{verbatim}
##          M       sd_M     min_M     max_M
## 1 0.258657 0.02391067 0.1944767 0.3897002
\end{verbatim}

\begin{itemize}
\tightlist
\item
  M = value at maximum density of distribution of M values.
\item
  sd\_M = standard deviation of M distribution.
\item
  min\_M = minimum value of M distribution.
\item
  max\_M = maximum value of M distribution.
\end{itemize}


\end{document}
